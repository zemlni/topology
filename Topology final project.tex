\documentclass[10pt]{article}
\title{Tree Project}
\author{Samuel Pease, Eva Arroyo, Nikita Zemlevskiy}
\date{5/4/2017}
\usepackage{graphicx}
\begin{document}
\maketitle
\newpage
\section*{Introduction}

Canopy architecture describes the surface of the tops of trees in a forest, how they overlay and grow over each other, how high they are, and what the shape of each individual tree is. Canopy architecture reflects disturbance history and species composition. These can be related to biomass, light availability, and disturbance (1).  The competition for light requires different species to niche differentiate and optimize for different amounts of light. In forests with high diversity of trees, there will (CITE) likely be high niche differentiation on all scales in order to live concurrently in the same forest. This will therefore be reflected in the canopy architecture, with more layers to the canopy. A mystery that has long puzzled ecologists is on the diversity of the tropical rainforest, and how so many species are able to live in the same forest (SEE IF YOU CAN RELATE THIS TO CANOPY ARCH OTHERWISE REMOVE). In addition, canopy architecture reflects disturbance history. If a canopy is disturbed in a significant way, such as through wind damage, this will likely cause tree toppling, which then creates a hole in the canopy. Though this hole will regrow, filled with other plants, the long term effects will be seen in the canopy architecture for up to 65 years afterwords (1). Disturbance history has profound effects on biomass, often secondary growth can result in a high net carbon sink, though decomposition will inevitably reverse this trend. In addition, anthropogenic changes can act as a large scale disturbance, which can similarly differentiate logged and unlogged forests for many years after the disturbance event. Canopy architecture is one of the summaries we have of complex forested ecosystems, and summarizing forest structure in a meaningful way can lead to understanding of how niche differentiation affects the whole forest.

The advent of LiDAR has allowed ecologists to remotely examine forest canopy heights . LiDAR functions by sending a burst of infrared light and recording the time it takes for light to return to the source, then using this to determine the distance the source is away from the detector. They can take several hundred samples in less than a second, making them extremely effective (CITE) By using this technology in planes flying over land surface, ecologists can obtain a very precise visualization of the canopy height across a forest, improving previous methods of assessing forest structure and size from ground transects. However, many methods using LiDAR are unable to make use of the topological data, simplifying it into maximum height and canopy size distribution across a forest, reducing data on a surface to data on a point. Persistence diagrams as a way of measuring both height and evenness, therefore have the potential to describe this surface of the canopy without reduction. Here we use four parks from the United States, to examine if persistence diagrams from LiDAR data can differentiate the different parks, and if these ecologically diverse sites with different disturbance regimes have significantly different canopy structure as analyzed with topological methods. We do not seek necessarily to explain differences in persistence across these forests, but to establish that persistence can differentiate these forests which should theoretically have vastly different canopy structures.


\section*{Data}
	We obtained LiDAR data from the National Ecological Observatory Network. The LiDAR dataset included interpretations of the raw waveform and point based returns into models of canopy height (CHMs), with just one tree height per pixel, which was what we used for our analyses. From the larger dataset of 13 ecological regions we imported LiDAR data from four sites Bartlett (BART), Great Smoky Mountains (GRSM), Harvard Forest (HARV), and Mountain Lake Biological Station (MLBS) from the Northeastern and Appalachian ecological regions on the eastern coast of the United States. These forests have significantly different forest structures and disturbance histories, from Harvard Forest, is a Northeastern experimental forest that has been selectively logged (CITE site); Mountain Lake Biological station, a forest in the Appalachian mountains of Virginia with history of chestnut blights (2); Great Smoky Mountains National Park, with a history of logging in the 1800s; and Bartlett forest which has selective logging. In addition these sites have vastly different species compositions, due to differences in soils and elevations, so we predict differences in canopy structure over all sites. These collectively resulted in 1050 CHMs upon which we ran our analyses.

\section*{Topological Methods}
To perform topological analysis and extract persistence diagrams from the data, two methods were used. The first of these methods was to obtain transects from the forest data. Transects are the canopy heights in one line across one a forest, creating a function along a line of canopy heights.These functions of one variable that mapped from x coordinate to height at that coordinate on the respective transect were generated for a random line across each CHM for each forest. Next, a standard sublevel-set filtration was performed on the function to create persistence diagrams. Zero dimensional persistence diagrams were obtained as a result of this method. Since the data was two dimensional by nature, we were interested how this method would compare to a 2 dimensional case, which led us to our second method.\\

The second method of obtaining persistence diagrams was to consider the data to be a function of two variables. For this we overlaid a grid over the forest and at each coordinate the function was equal to the height of the canopy there. Next, we took a 2 dimensional sublevel-set filtration of the function. By sweeping from the minimum height to the maximum height, we were able to obtain persistence diagrams for the data. To accomplish this task, we used the code written by Dr. Nate Strawn provided by Dr. Chris Tralie. Using this method, zero and one dimensional persistence diagrams of this representation of the data were obtained. Once we had generated persistence diagrams from all of our data we converted them into graphs of persistence plotted against birthtime. We summarize the persistence diagrams for each site by reporting the maximum persistence across all CHMs, the mean maximum persistence for each persistence diagram in the data set for the site, and the mean of the mean persistence for each persistence diagram in each site.

\section*{Machine Learning Methods}

From the persistence diagrams, we overlaid a grid and counted the number of points that fell within each grid cell in order to generate vectors. For each type of persistence we first found the longest persistence across all diagrams and use this as the maximum value for all grids in order to generate equal-length vectors. The grids went from zero to this maximum value by ones.

Next we used these high dimensional vectors to do machine learning through SVM. SVM learning or support vector machine learning is a process of finding the hyperplane that best separates data points by category. However there are many examples of data that is clearly separable but not simply by a hyperplane, such as data easily separated by a circle. In these cases a kernel function is used. This is done by raising the dimension of the data so that a hyperplane does accurately separate the categories of data. In order to determine which kernel was the most effective at separating data we used cross validation. The data was split up into training ($80\%$ of the data in all cases) and validation data so that we could test our classification on different vectors than those used to generate the separating hyper plane. The classification was then tested on the testing data using the different kernel functions and different cost and gamma values as the parameters of the classification. The optimal kernel function, cast, and gamma values were then selected to produce a hyperplane that best separated the data by category. To do this we used code from Deepanshu Bhalla.

We ran the SVM to classify all four of our forests. This classifies each of the four variables against every other variable and then using a voting technique to achieve the best overall prediction model. We report all sites with Kolmogorov-Smirnof (KS) tests for goodness of fit of our SVMs, and area under the receiver operating curves for accuracy of prediction.

\section*{Results}
Transects

We calculated transect data for four sites and a total of 224 persistence diagrams. Across sites, seven had to be removed, which resulted in blank persistence diagrams. The highest persistence is for the Great Smoky Mountains as well as the highest mean maximum persistence (Table 1). The calculated area under the curve was 0.610 (Figure 1), and a KS of 0.242.

Table 1: Summary statistics across all transects in a specific site with mean and standard deviation in parenthesis.

\begin{table}[]
\centering
\caption{My caption}
\label{Table 1: Summary statistics across all transects in a specific site with mean and standard deviation in parenthesis.}
\begin{tabular}{|l|l|l|l|l|}
\hline
                         & BART          & GRSM          & HARV          & MLBS          \\ \hline
Mean Maximum Persistence & 22.18 (2.907) & 29.75 (6.845) & 23.21 (3.141) & 24.71 (5.240) \\ \hline
Maximum Persistence      & 31            & 49            & 38            & 39            \\ \hline
Mean Mean Persistence    & 7.610 (1.799) & 8.453 (2.005) & 5.869 (3.090) & 7.156 (2.452) \\ \hline
\end{tabular}
\end{table}


\includegraphics[scale = 0.05]{bartlett_transects_0d_persistence}\\
\includegraphics[scale = 0.05]{harvard_transects_0d_persistence}\\
\includegraphics[scale = 0.05]{grsm_transects_0d_persistence}\\
\includegraphics[scale = 0.05]{mlbs_transects_0d_persistence}\\
\includegraphics[scale = 0.05]{transect_roc}\\

Figure 1: False positives versus true positives for prediction of site 1


0d Persistance

We analyzed a total of 400 canopy height models over the four sites to examine the zero dimensional persistence. This resulted in an AUC of .677 and a KS of .312 (Figure 2).






Table 2: Summary statistics across all transects in a specific site with mean and standard deviation in brackets




Figure 2: True positive by false positives for the 0D persistence of the Canopy Height Models

1d Persistance
We used the same set of 400 canopy height models for the one dimensional persistence. This resulted in a AUC of .815 and a KS of .5 (Figure 3). The highest average maximum persistence across sites was at GRSM, which also had the highest maximum persistence (Table 3).






Table 3: Summary statistics across all transects in a specific site with mean and standard deviation in brackets


BART
GRSM
HARV
MLBS
Mean maximum Persistence
23.84 (1.134)
40.91 (3.528)
28.61 (2.112)
28.71 (1.701)
Maximum persistence
27
52
37
36
Mean mean persistence
9.0177 (.572)
15.068 (1.323)
12.445 (.669)
11.444 (.492)



Figure 3: True positive by false positives for the 1-D persistence of the Canopy Height Models

Discussion
	The data resource that was used in this study contained immense amounts of data. This study could be extended to any number of forest locations, following a similar methodology that was outlined in this study. The data resource also contains numerous other attributes measured across the forests, not just canopy height information. Examples of these attributes include soil type information, logging information and other information. Including variables such as those into the classification model could prove beneficial, and may create a better predictor. In addition, it would be useful to classify on these additional parameters, not just on forest type/location. Such a model could be useful for various reasons. One may be interested in finding out whether a forest was logged or not, or what kind of soil a forest most likely has without actually sampling the forest. This would be possible from these additional models. Other classification methods, such as k nearest nearest neighbors may prove to be more effective, and a logical extension of this study would be to compare the performance of different learning methods.

\section*{Bibliography}
Bhalla, D. (2017 ). Support Vector Machine Simplified using R. Retrieved May 02, 2017, from http://www.listendata.com/2017/01/support-vector-machine-in-r-tutorial.html
Strawn, N. (n.d.). MORSEFILTRATION2D. Retrieved from https://d1b10bmlvqabco.cloudfront.net/attach/j1m8suensxh6ez/ho32xurye2y7nv/j1o35t5rkuhu/morseFiltration2D.m.

\end{document}
